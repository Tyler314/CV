\documentclass{resume}

\begin{document}
\begin{center}
{\LARGE \bf Tyler Roberts} \\[1mm]
\footnotesize
(262) 388-4050 $|$
tjroberts314@gmail.com $|$
\href{https://www.linkedin.com/in/tyleroberts}{https://www.linkedin.com/in/tyleroberts} $|$
\href{http://tyler.engineering}{http://tyler.engineering}
\end{center}
\begin{flushleft}

{\textbf{\large Summary}} \\
{
\footnotesize
\tab Enthusiastic computer engineer with 4+ years experience in software development and design. Extensive work with Python, including the improvement of an application used in cutting edge scientific research at the University of Wisconsin - Madison, the development of an application that efficiently crawled through Micron's documentation and translated language syntax, and broad work with the SciPy package within academia. Considerable development with C and C++ in embedded systems and lower level programming, as well as exposure to computer architecture and circuit design. Very interested in systems and how they all interact with one another. \\[2mm]
}
\underline{Computer Skills} - Python, C++, C, Linux, Bash, Verilog, SystemVerilog, Matlab, Java, Git. \\[5mm]
\longunderline{\textbf{\large Experience }}\\[2mm]

\textbf{IBM Corporation}, Yorktown Heights, NY \hfill July 2017 - Present\\
\textbf{Software Developer, Watson Health Cloud}\\
{\footnotesize
	\ttab- Increased test case coverage from 30\% to 90\% using Python unit tests since joining the team. \\
	\ttab- Developed Python application on another team that moved their locally stored data and user accounts to Bluemix,
	\ttab IBM's cloud platform as a service. \\
	\ttab- Designed and engineered a web application from the ground up using HTML, CSS, Angular, and Bootstrap. \\
	\ttab Responsible for the front-end, including page navigation using a MVC pattern.  Worked closely with two others as they
	\ttab developed the other layers of the application. \\
	\ttab - Used Docker to rigorously test the deployment of the web application onto a server. \\
	\ttab- \textit{Tools \& Technologies: Python (2.7), Docker, HTML, Angular, Javascript, Linux, MacOS.}\\[3mm]
}

\textbf{UW-Madison Plasma Physics Dept.}, Madison, WI \hfill Jan. 2016 - May 2017 \\
\textbf{Scientific Programmer}\\
{\footnotesize
	\ttab- Inherited former Ph.D. candidate's Python application, and enhanced it to communicate with additional instrumentation
	\ttab added to the experiment. \\
	\ttab- Wrote new code to parse binary data recorded from experiments, and store it to a database. \\
	\ttab- Wrote a GUI using tkinter to improve productivity for the team as they used the application for their research. \\
	\ttab- Collaborated extensively with scientists and professors to deliver a fully functional application for their research. \\
	\ttab- Organized and taught Python tutorials for graduate students, professors, and scientists unfamiliar with Python and OOP. \\
	\ttab- \textit{Tools \& Technologies: Python (2.7), C++,  Matlab, MacOS, Linux.}\\[5mm]
}

\textbf{Intel Corporation}, Hillsboro, OR \hfill May 2016 - Aug. 2016\\
\textbf{Pre-Silicon Validation Engineering Intern}\\
{\footnotesize
\ttab- Improved debug tool by creating my own checkers and algorithms that were used to validate the SoC architecture. My \ttab improvements were able to detect and isolate several bugs found within the design.\\
\ttab- Developed Python modules in large code base for validation teams to share key architectural, test and debug knowledge. \\
\ttab- Enhanced a validation tool by developing features that created easy debug for members of the design team. Managed to \ttab increase the productivity of the developers and validators as well as save time for the company. \\
\ttab- Collaborated with several teams within DDG to determine the best way to provide feedback in the debugging process. \\
\ttab- \textit{Tools \& Technologies: Python (3.4), Perl, SystemVerilog, OVM/UVM, Unix.}\\[3mm]
}

\textbf{Micron Technology}, Longmont, CO \hfill May 2015 - Aug. 2015\\
\textbf{Product Validation Engineering Intern}\\
{\footnotesize
\ttab- Wrote python program that manipulated Micron's script documentation to convert Sphinx docs to Pydoc docs. \\ \ttab Application allowed user to browse through documentation, and translated the pages in real time.\\
\ttab- Tested solid state drives to ensure they performed correctly when given certain commands.\\
\ttab- Worked with Micron's test automation platform, and used FIO, an I/O benchmarking tool, for testing the SSDs.\\
\ttab- \textit{Tools \& Technologies: Python (2.7), Bash, Linux, Git, JIRA, Jenkins.}\\[5mm]
}

\longunderline{\textbf{\large Education}} \\
{\bigsize
	B.S. Computer Engineering, Computer Science, \& Mathematics with Physics Certificate (minor). \\
	University of Wisconsin - Madison, May 2017 } \\[5mm]

%\textbf{UW-Madison Plasma Physics Dept.}, Madison, WI \hfill Jan. 2013 - Present \\
%\textbf{Mechanic's Assistant}\\
%{\footnotesize
%\ttab- Work directly under the supervision of the mechanical engineer of the department.\\
%\ttab- Assist with mechanical issues and restoration of the equipment, work consistently with heavy %machinery.\\[10mm]
%}




%\longunderline{\textbf{\large Projects}} \\[2mm]
%\textbf{Take Data 3 - UW-Madison Plasma Physics} \\%\hfill July - Aug. 2015 \\
%{\footnotesize
%	\ttab- Improved Ph.D. candidate's code so that it would communicate with additional instrumentation added to the experiment. \\
%	\ttab - Collaborated extensively with scientists and professors to deliver a fully functional tool for their research. \\
%	\ttab - Taught new graduate students Python and the code base so they could carry on my work where needed. 
%}
%\\[2mm]
%\textbf{Debugging Tool - Intel Corporation} \\%\hfill July 2016 \\
%{\footnotesize
	%\ttab- Improved upon a post-silicon debugging tool in pre-silicon by adding additional checkers and algorithms for signal detection.\\
	%\ttab- Caught several bugs in the SoC and helped the designers to fix the bugs faster and more effectively. \\
	%\ttab- Added improved debug hints and documentation for debugging tool to improve efficiency of debugging across teams. %\\[2mm]
%}
%\textbf{Validation Tool - Intel Corporation} \\
%{\footnotesize
%	\ttab- Developed Python modules in large code base for validation teams to share key architectural, test and debug knowledge. \\
%	\ttab- Collaborated with several teams within DDG to determine the best way to provide feedback in the debugging process. \\
%	\ttab -  Developed features that created easy debug and enabled members of the design team to become effective debuggers. \\[4mm]
%}

%\textbf{Python Pydoc Wrapper - Micron Technology} \\%\hfill July - Aug. 2015 \\
%{\footnotesize
%\ttab- Wrote python program that manipulated Micron's script documentation to convert Sphinx docs %to Pydoc docs.\\
%\ttab- Overide Pydoc methods, and implemented my own class, methods and algorithms to complete %the task.\\
%\ttab- Creates html pages of the documentation with links to other test documentations. \\[5mm]
%}

%	\textbf{GitHub}:
% \href{https://github.com/Tyler314}{https://github.com/Tyler314} \\[3mm]

%\longunderline{\textbf{\large Volunteer}} \\[2mm]
%\textbf{Greater University Tutoring Service}, Madison, WI \hfill Feb. 2015 - May 2015 \\

%\textbf{Tutor for Calculus \& Trigonometry} \\
%{\footnotesize
%\ttab- Tutored a group of 5 students once per week, with each tutoring session lasting 2 hours. \\
%\ttab- Prepared for class by reviewing lecture material, creating example problems, and preparing detailed notes. \\
%\ttab- Nominated as one of the top tutors in the program at the end of the semester. \\[2mm]
%}
%\textbf{Final Project - Intro to Microprocessors (ECE 353)} \hfill May 2015\\
%- Designed a turn based fighting game entirely in C, on a Texas Instruments microcontroller.\\
%- Utilized peripherals such as the GPIO pins, joystick, LCD screen, Nordic Wireless Radio, and the UART; all of which had to be configured in the C code.\\[3mm]

%\textbf{LinkedIn}: %\href{https://www.linkedin.com/in/tyleroberts}{https://www.linkedin.com/in/tyleroberts}



\end{flushleft}



























\end{document}

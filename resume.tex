\documentclass{resume}

\begin{document}
\begin{adjustwidth}{75mm}{}
{\LARGE \bf Tyler Roberts} \hfill 15 Bank St., Apt. 116B
\end{adjustwidth}
\begin{adjustwidth}{80mm}{}
{\normalsize tjroberts314@gmail.com} \hfill White Plains, NY 10606
\end{adjustwidth}
\begin{adjustwidth}{87.5mm}{}
{\normalsize (262) 388-4050}
\end{adjustwidth}
\begin{flushleft}
	\textbf{Objective} - Full-Time position in Computer Engineering. \\
	\textbf{LinkedIn}:
	\href{https://www.linkedin.com/in/tyleroberts}{https://www.linkedin.com/in/tyleroberts} \\
	\textbf{Website}:
	\href{http://tyler.engineering}{http://tyler.engineering} \vspace{\baselineskip}

\longunderline{\textbf{\large Education}} \\
{\bigsize
B.S. Computer Engineering, Computer Science, \& Mathematics with Physics Certificate. \\
University of Wisconsin - Madison \\
\underline{Computer Skills} - Python, C++, C, Linux, Bash, Verilog, SystemVerilog, Matlab, Java, Git.} \\[5mm]
\longunderline{\textbf{\large Experience }}\\[2mm]

\textbf{IBM Corporation}, Yorktown Heights, NY \hfill July 2017 - Present\\
\textbf{Software Engineer, Watson Health Cloud}\\
{\footnotesize
	\ttab- Work on projects that regulate patient data by adhering to data standards within health care.\\
	\ttab- Unit test and verify deployment of code across many environments and servers. \\
	\ttab- Worked on front end UI for our clients to interact and use our services. \\
	\ttab- \textit{Tools \& Technologies: Python (2.7), Java, Docker, HTML, Angular, Linux, MacOS.}\\[3mm]
}

\textbf{UW-Madison Plasma Physics Dept.}, Madison, WI \hfill Jan. 2016 - May 2017 \\
\textbf{Scientific Programmer}\\
{\footnotesize
	\ttab- Collaborated with scientists and professors by helping them with code used in their research. \\
	\ttab- Worked with code in Python and C++ used in the Madison Symmetric Torus experiment. \\
	\ttab- Organized and taught Python tutorials for grad students and postdocs unfamiliar with Python and OOP. \\
	\ttab- \textit{Tools \& Technologies: Python (2.7), C++,  MacOS, Linux.}\\[5mm]
}

\textbf{Intel Corporation}, Hillsboro, OR \hfill May 2016 - Aug. 2016\\
\textbf{Pre-Silicon Validation Engineering Intern}\\
{\footnotesize
\ttab- Improved debug tool by creating my own checkers and algorithms that are used to validate the SoC architecture. My \ttab improvements were able to detect and isolate several bugs found within the design.\\
\ttab- Enhanced a validation tool by developing features that created easy debug for members of the design team. Managed to \ttab increase the productivity of the developers and validators as well as save time for the company. \\
\ttab- \textit{Tools \& Technologies: Python (3.4), Perl, SystemVerilog, OVM/UVM, Unix.}\\[3mm]
}

\textbf{Micron Technology}, Longmont, CO \hfill May 2015 - Aug. 2015\\
\textbf{Product Validation Engineering Intern}\\
{\footnotesize
\ttab- Tested solid state drives to ensure they performed correctly when given certain commands.\\
\ttab- Wrote a python script that served as a wrapper for the company's code documentation, now used at Micron.\\
\ttab- Worked with Micron's test automation platform, and used FIO, an I/O benchmarking tool, for testing the SSDs.\\
\ttab- \textit{Tools \& Technologies: Python (2.7), Bash, Linux, Git, JIRA, Jenkins.}\\[3mm]
}

%\textbf{UW-Madison Plasma Physics Dept.}, Madison, WI \hfill Jan. 2013 - Present \\
%\textbf{Mechanic's Assistant}\\
%{\footnotesize
%\ttab- Work directly under the supervision of the mechanical engineer of the department.\\
%\ttab- Assist with mechanical issues and restoration of the equipment, work consistently with heavy %machinery.\\[10mm]
%}

\longunderline{\textbf{\large Projects}} \\[2mm]
\textbf{Take Data 3 - UW-Madison Plasma Physics} \\%\hfill July - Aug. 2015 \\
{\footnotesize
	\ttab- Improved Ph.D. candidate's code so that it would communicate with additional instrumentation added to the experiment. \\
	\ttab - Collaborated extensively with scientists and professors to deliver a fully functional tool for their research. \\
	\ttab - Taught new graduate students Python and the code base so they could carry on my work where needed. 
}
\\[2mm]
\textbf{Debugging Tool - Intel Corporation} \\%\hfill July 2016 \\
{\footnotesize
	\ttab- Improved upon a post-silicon debugging tool in pre-silicon by adding additional checkers and algorithms for signal detection.\\
	\ttab- Caught several bugs in the SoC and helped the designers to fix the bugs faster and more effectively. \\
	\ttab- Added improved debug hints and documentation for debugging tool to improve efficiency of debugging across teams. \\[2mm]
}
\textbf{Validation Tool - Intel Corporation} \\
{\footnotesize
	\ttab- Developed Python modules in large code base for validation teams to share key architectural, test and debug knowledge. \\
	\ttab- Collaborated with several teams within DDG to determine the best way to provide feedback in the debugging process. \\
	\ttab -  Developed features that created easy debug and enabled members of the design team to become effective debuggers. \\[4mm]
}

%\textbf{Python Pydoc Wrapper - Micron Technology} \\%\hfill July - Aug. 2015 \\
%{\footnotesize
%\ttab- Wrote python program that manipulated Micron's script documentation to convert Sphinx docs %to Pydoc docs.\\
%\ttab- Overide Pydoc methods, and implemented my own class, methods and algorithms to complete %the task.\\
%\ttab- Creates html pages of the documentation with links to other test documentations. \\[5mm]
%}

%	\textbf{GitHub}:
% \href{https://github.com/Tyler314}{https://github.com/Tyler314} \\[3mm]

%\longunderline{\textbf{\large Volunteer}} \\[2mm]
%\textbf{Greater University Tutoring Service}, Madison, WI \hfill Feb. 2015 - May 2015 \\

%\textbf{Tutor for Calculus \& Trigonometry} \\
%{\footnotesize
%\ttab- Tutored a group of 5 students once per week, with each tutoring session lasting 2 hours. \\
%\ttab- Prepared for class by reviewing lecture material, creating example problems, and preparing detailed notes. \\
%\ttab- Nominated as one of the top tutors in the program at the end of the semester. \\[2mm]
%}
%\textbf{Final Project - Intro to Microprocessors (ECE 353)} \hfill May 2015\\
%- Designed a turn based fighting game entirely in C, on a Texas Instruments microcontroller.\\
%- Utilized peripherals such as the GPIO pins, joystick, LCD screen, Nordic Wireless Radio, and the UART; all of which had to be configured in the C code.\\[3mm]

%\textbf{LinkedIn}: %\href{https://www.linkedin.com/in/tyleroberts}{https://www.linkedin.com/in/tyleroberts}



\end{flushleft}



























\end{document}
